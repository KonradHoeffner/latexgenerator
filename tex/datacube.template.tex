\documentclass[tikz]{standalone}
\usetikzlibrary{matrix,paths.ortho,calc,positioning-plus}
\tikzset{local cs/.style n args=4{shift={(#1)}, x=($(#2)-(#1)$), xslant={#3}, yscale={#4}}}
\tikzset{if/.code n args=3{\pgfmathparse{#1}\ifnum\pgfmathresult=0 
  \pgfkeysalso{#3}\else\pgfkeysalso{#2}\fi}}

\begin{document}
\begin{tikzpicture}[
  tight matrix/.style={
    matrix of nodes, inner sep=+0pt, outer sep=+0pt,
    every cell/.append style={
      every node/.append style={
        outer sep=+0pt,
        inner xsep=+.1em,
        inner ysep=+.3333em, % default (overwritten by matrix inner sep)
        align=center,
        text depth=+0pt, % depth("y"),
        text height=height("M"),
        text width=width("MMM-00-00"), % possibly other approaches
      }}},
  desc/.style={/utils/exec=\scriptsize}, % font key is not perfect
  continent/.style={
    desc, align=center, anchor=east},
  hemi/.style={desc, align=center, text width=width("Hemisphere")},
  route/.style={desc},
  Route/.style={font=\bfseries},
]

% The matrix
\matrix[tight matrix, draw] (m) {
§MATRIX§
};

% The 3D
\tikzset{my cs/.style={local cs={m.north west}{m.north east}{1}{0.35*§LAYERS§}}}
\begin{scope}[my cs]
% upper left z axis and upper x axis in the distance
\draw (0,0) |- (1,1) -- (right:1)
    (m.south east) -- ++ (up:1) -- (1,1);

% The lines

% vertical inside lines 
\foreach \c in {1,...,§COLUMNS-1§}
  \draw (m-§ROWS*MEASURES§-\c.south east) -- (m-1-\c.north east) -- ++ (up:1);

% horizontal inside lines, wrapping into the distance to the right
\foreach \c in {1,...,§ROWS*MEASURES-1§}
  \draw[if={isodd(\c)}{densely dashed}{}]
    (m-\c-1.south west) -- (m-\c-§COLUMNS§.south east) -- ++(up:1);

% horizontal lines on top and vertical lines to the right in the distance
\foreach \c in {1,...,§LAYERS§}
  \draw%[if={isodd(\c)}{densely dashed}{}]
    (up:\c/§LAYERS§) coordinate (tl-\c) -- ++ (right:1) -- ([shift=(up:\c/§LAYERS§)]m.south east);

\end{scope}

% The Descriptions

% y dimension values
\foreach \st/\lt[count=\c from 0, evaluate={\d=int(2*\c+1)}]
 in {§YVALUES§}
   \node[continent] (\st) at (m-\d-1.south west) {\lt};

% z dimension values
\begin{scope}[node distance=.25cm, hvvh/distance=.2cm, my cs]
\foreach \st[count=\c from 0, evaluate={\d=int(\c+1)}] in {§ZVALUES§}
  \node[continent, left=of tl-\d](\st){\st};
\end{scope}

\end{tikzpicture}
\end{document}
